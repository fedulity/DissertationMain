\chapter{Разработка программных модулей системы дистанционного обучения и результаты экспериментов} \label{chapt4}

\section{Архитектура системы электронного обучения на основе онтологий} \label{sect4_1}

Сервер системы дистанционного обучения ECOLE основа на платформе Information Workbench[15]. Платформа Information Workbench предоставляет функционал для работы с открытыми связными данными Linked Open Data. Платформа основана на использовании программных модулей с открытым исходным кодом. Пользовательский интерфейс сервера ECOLE основан на модуле семантической разметки Semantic MediaWiki[16]. Данный модуль позволяет использовать переопределенные шаблоны и визуальные средства для отображения семантических данных в виде Wiki-страниц. Редактирование и управление RDF данными системы реализовано с использованием платформы OpenRDF Sesame. Сервер системы ECOLE поддерживает запросы SPARQL. Сервер предоставляет открытую точку доступа для SPARQL запросов.

Внешним интерфейсом системы дистанционного обучения ECOLE является облегченная система управления образованием Learning Management System (LMS). LMS предназначена для удобного представления учебных материалов пользователям системы. LMS обладает локальным хранилищем и производит управление пользовательскими данными, настройками и результатами обучения студентов. Внешний интерфейс системы предоставляет функционал по администрированию системы и управлению доступом к данным системы. В LMS реализованы модули для отображения видео-лекций, слайдов, тестов и практических заданий.

Внешний интерфейс взаимодействует с сервером системы с помощью запросов к открытой точке доступа SPARQL. Внешний интерфейс получает с сервера данные по учебным материалам и отношения между объектами курса. Приватные персональные данные пользователей и настройки LMS хранятся в локальной памяти внешнего интерфейса. Общая архитектура системы дистанционного обучения ECOLE представлена на рисунке \ref{img:overall_arch}.

\begin{figure} [h] 
  \center
  \includegraphics [scale=0.6] {overall_arch}
\caption{Общая архитектура системы дистанционного обучения ECOLE.}
  \label{img:overall_arch}  
\end{figure}

LMS реализована на языке Python с использованием Django Web Framework[17]. Библиотека SPARQLWrapper использована отправки запросов к точке доступа SPARQL. Когда пользователь завершает тест LMS собирает результаты теста, ответы на задания и дополнительную статистику и записывает полученные данные на сервер с помощью запроса SPARQL Update Query[18]. В результате прохождения теста студентом на сервере создается объект класса <<AttemptToPassTest>> с набором правильных и неправильных ответов на задания теста. Данный объект связывается с объектом студента. В целях безопасности персональных данных объекты студентов идентифицируются с использованием хеш-суммы электронной почты студента.

После прохождения теста система предоставляет студенту информацию о количестве и доле правильных ответов на задания теста. Так же студенту предоставляется список предметных терминов для повторения. Система генерирует список проблемных терминов для студента, используя результаты теста и связи между предметными терминами системы и заданиями теста. Для каждого предметного термина связанного с заданиями теста производится расчет рейтинга на основе ответов студента. Лист проблемных терминов сортируется в порядке возрастания рейтинга. Чем выше рейтинг термина, тем больше правильных ответов дал студент на задания связанные с данным терминов и тем меньше затруднений вызвал у студента данный термин. Рейтинг термина, полученный в контексте прохождения теста, может быть отражен в глобальном рейтинге знаний терминов студентом. Глобальный рейтинг знаний для каждого термина позволяет студенту выявлять проблемные термины и восполнять знания по ним.   


\section{Описание пользовательских интерфейсов разработанной системы электронного обучения} \label{sect4_2}

%%%%%%%

%% СКРИНШОТЫ И ОПИСАНИЕ СИСТЕМЫ

%%%%%%%

\section{Метод преобразования онтологий системы дистанционного обучения в формат SCORM} \label{sect4_3}

%%%%%%%

%% ПЕРЕВЕСТИ СТАТЬЮ

%%%%%%%

\section{Результаты применения методики агрегации данных в системе электронного обучения} \label{sect4_4}

Основной набор данных системы дистанционного обучения ECOLE формировался вручную. Часть данных была создана с помощью методов наполнения онтологии описанных в главе 3. Набор данных системы состоит из объектов образовательного процесса, таких как курс, модуль, лекция, тест, практика, предметный термин, предметная область и книга. В результате работы NLP  алгоритмов по извлечению предметных терминов из тестов были получены результаты, представленные в таблице \ref{table:nlp_resutls}. С одной стороны с терминами было связано 95\% заданий тестов. С другой стороны более 50\% терминов курса остались не связанными с заданиями тестов. Одной из причин данного явления является косвенное употребление предметных терминов в задании. Чтобы решить такое задание необходимо знать предметный термин, который не упомянут ни в тексте задания, ни в тексте ответов. Примером таких заданий являются задачи на поиск длинны гипотенузы треугольника при известных катетах. Косвенно связанным предметным терминов в данном примере является термин «Теорема Пифагора». В текущей реализации метода извлечение косвенных терминов не производится. В будущем планируется использование семантических связей между терминами для выявления косвенных предметных терминов в заданиях тестов.

\begin{table}[h!]
\centering
\caption{Результаты работы алгоритмов обработки естественного языка по извлечению предметных терминов из тестов.)}
\label{table:nlp_resutls}
\begin{tabular}{ |p{12cm}|c|  }
\hline Количество обработанных заданий & 20 \\
\hline Процент связанных заданий, \% & 95 \\
\hline Процент несвязанных заданий, \% & 5 \\
\hline Количество извлеченных терминов-кандидатов & 155 \\
\hline Количество терминов извлеченных вручную & 30 \\
\hline Термины системы, совпавшие с терминами-кандидатами, \% & 50 \\
\hline Термины-кандидаты, совпавшие с терминами системы, \% & 8 \\
\hline Термины-кандидаты, добавленные в систему после прохождения проверки, \%  & 6 \\
\hline Ложные термины-кандидаты, \% & 86 \\
\hline
\end{tabular}
\end{table}  


\section{Результаты применения методов анализа на учебных курса и группах студентов} \label{sect4_5}

Модуль автоматического расчета оценки и рейтинга знаний студента по предметным терминам и областям был разработан на языке Python и интегрирован в систему электронного обучения ECOLE с использованием Django Web Framework. Работа модуля была применена при прохождении студентами курса <<Интеллектуальные системы>>. В результате работы модуля было рассчитано значение для предметных терминов из области <<Экспертные системы>>. В данный момент в предметной области содержится 43 термина. Десять терминов с наибольшим значением представлены в таблице \ref{table:term_importance_result}. Полученные результаты демонстрируют значение предметных терминов в образовательном процессе. Для изучения предметных терминов одной области могут потребоваться знания терминов другой предметной области. Таким образом, значения терминов не зависят от предметных областей, к которым данные термины относятся. Расчеты показывают, что базовые понятия в предметной области не являются самыми важными в ней.    

\begin{table}
\centering
\caption{Значения предметных терминов области <<Экспертные системы>>}
\label{table:term_importance_result}
\begin{tabular}{|p{7cm}|c|}
\hline Предметный термин & Значение \\
\hline Инженерия знаний & 3.749703 \\
\hline Знания & 1.796706 \\
\hline Извлечение знаний & 1.796706 \\
\hline Рабочая память & 1.706427 \\
\hline Представление (репрезентация) знаний & 1.606853 \\
\hline Процесс разрешения конфликтов & 1.374309 \\
\hline Язык представления знаний & 1.191552 \\
\hline Экстенсионал & 1.169817 \\
\hline Интенсионал & 1.169817 \\
\hline Интуитивные знания & 1.169817 \\
\hline
\end{tabular}
\end{table}

При прохождении студентом  лекций и тестов модуль рассчитывает рейтинги предметных терминов и областей в соответствии с разработанными методами. В результате студенту демонстрируется список предметных областей и терминов с оценками. Интерфейс списка оценок представлен на рисунке \ref{fig:user_screen_result}.

\begin{figure} [h] 
  \center
  \includegraphics [scale=0.45] {user_screen_result}
  \caption {Интерфейс списка предметных областей и терминов с рейтингами знаний в системе ECOLE} 
  \label{fig:user_screen_result}
\end{figure}


%%%%%%%%

% НУЖНЫ РЕЗУЛЬТАТЫ ПО ОПРОСАМ СТУДЕНТОВ

%%%%%%%%

\clearpage