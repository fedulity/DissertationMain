\chapter{Использование семантических технологий и распределенных систем в образовательном процессе} \label{chapt1}




\section{Характеристики и задачи распределенных систем} \label{sect1_1}

Распределенная система — это набор независимых компьютеров, представляющийся их пользователям единой объединенной системой.(Э. Таненбаум, 2003)

Одним из примеров  распределенной системы является сеть World Wide Web. Данная система предоставляет простую, целостную и единообразную модель распределенных документов. Чтобы увидеть документ, пользователю достаточно активизировать ссылку. После этого документ появляется на экране. 

Одно из главных требований к распределенной системе это ее единство в представлении пользователя. Применение распределенных систем стало актуальным при значительном увеличении скорости передачи данных между независимыми компьютерами и возникновении необходимости проведения трудоемких вычислительных задач. Распределенные системы в первую очередь заменяют дорогостоящие суперкомпьютеры (Ю.И. Шокин, 1998). Главным преимуществом системы из независимых компьютеров является возможность ее масштабирования путем добавления или удаления из системы новых машин. Такие системы позволяют производить трудоемкие вычисления, объединять удаленные банки знаний в единое облако и предоставлять необходимые услуги пользователям, скрывая от них свое реальное местоположение (H. Wang, 2010).

Распределенная система обладает рядом характеристик. Первая из таких характеристик состоит в том, что от пользователей скрыты различия между компьютерами и способы связи между ними. То же самое относится и к внешней организации распределенных систем. Другой важной характеристикой распределенных систем является способ, при помощи которого пользователи и приложения единообразно работают в распределенных системах, независимо от того, где и когда происходит их взаимодействие. Распределенные системы должны также относительно легко поддаваться расширению, или масштабированию (G.F. Coulouris, 2009).  Эта характеристика является прямым следствием наличия независимых компьютеров, но в то же время не указывает, каким образом эти компьютеры на самом деле объединяются в единую систему. Распределенные системы обычно существуют постоянно, однако некоторые их части могут временно выходить из строя. Пользователи и приложения не должны уведомляться о том, что эти части заменены или починены или что добавлены новые части для поддержки дополнительных пользователей или приложений.

Основными задачами распределенных систем являются:

\begin{itemize}
\item соединение пользователей с ресурсами. Распределенная система должна облегчить пользователям доступ к удаленным ресурсам и обеспечить их совместное использование, регулируя этот процесс (S. Ghosh, 2010). Ресурсы могут быть виртуальными, однако традиционно они включают в себя принтеры, компьютеры, устройства хранения данных, файлы и данные. Web-страницы и сети также входят в этот список. 
\item скрытие факта, что процессы и ресурсы физически распределены по множеству компьютеров. Распределенные системы, которые представляются пользователям и приложениям в виде единой компьютерной системы, называются прозрачными.
\item предоставление служб со стандартным синтаксисом и семантикой. В распределенных системах службы обычно определяются через интерфейсы, которые часто описываются при помощи языка определения интерфейсов.
\item обеспечение масштабируемости системы. Масштабируемость системы может измеряться по трем различным показателям. Во-первых, система может быть масштабируемой по отношению к ее размеру, что означает легкость подключения к ней дополнительных пользователей и ресурсов. Во-вторых, система может масштабироваться географически, то есть пользователи и ресурсы могут быть разнесены в пространстве. В-третьих система может быть масштабируемой в административном смысле, то есть быть проста в управлении при работе во множестве административно независимых организаций. Система, обладающая масштабируемостью по одному или нескольким из этих параметров, при масштабировании часто дает потерю производительности.
\end{itemize}

Сама распределенная система является лишь фундаментом для реализации более сложных сетевых и системных взаимодействий. Внутри распределенной системы могут быть описаны механизмы обмена данными, поддержания работоспособности системы (P. Jalote, 1994), распределения нагрузки на систему (B.A. Shirazi, 1995) и безопасности передачи данных внутри системы (W.Y. Chen, 2011). 

Примером реализации системы основанной на принципах и механизмах распределенной системы является система дистанционного обучения. Система дистанционного обучения решает с помощью механизмов распределенной системы задачи по масштабируемости системы и представления системы в виде единой службы. 


\section{Классификация систем дистанционного обучения и их развитие в России и за рубежом} \label{sect1_2}

Система дистанционного обучения - система управления учебной деятельностью. Позволяет разрабатывать, управлять и распространять учебные онлайн-материалы с обеспечением совместного доступа. Данный вид систем предназначен для проведения удаленного дистанционного образовательного процесса включающего в себя процесс получения знаний и процесс проверки знаний пользователя (D. Keegan, 1996).

Существует несколько вариантов использования технологий дистанционного обучения:

\begin{itemize}
\item в качестве дополнительной поддержки основного курса обучения (здесь технологиям ДО отводится вспомогательная роль),
\item в качестве основы для самообразования (в этом случае учащиеся самостоятельно приобретают и осваивают готовые электронные образовательные продукты, например - мультимедиа курсы),
\item в качестве основной образовательной технологии (Г.В. Можаева 2000). В этом случае создается постоянная группа учащихся в периферийном центре, которая работает под руководством и под контролем педагога или координатора. Он контролирует ход учебного процесса, своевременное выполнение заданий учащимися, консультирует, помогает учащимся в процессе освоения курса.
\end{itemize}

На данный момент существует два вида дистанционного обучения: 

\begin{itemize}
\item асинхронное дистанционное обучение, которое предусматривает, что ученик сам определяет темп своего обучения. При этом обучающийся имеет выбор между различными носителями информации, ученик может выполнять задания в соответствии с аудиторной программой или планом, а затем передавать готовую работу преподавателю для оценки.
\item синхронное дистанционное обучение (Ю.А. Айзина, 2009). Данный вид предусматривает общение учеников и преподавателей в реальном времени через виртуальные аудитории. 
\end{itemize}

Историю развития дистанционного обучения можно разделить на несколько поколений:

\begin{itemize}
\item 1-ое поколение. Технические средства, характеризующиеся отсутствием интерактивности (радио или аудио кассеты, учебники, посланные студентам с минимальным общением по телефону).
\item 2-ое поколение. Асинхронно интерактивные курсы, характеризующиеся трансляциями (телевидение или радио) с призывом к интерактивности (в течение или после) либо по телефону, либо по электронной почте.
\item 3-е поколение. Характеризуется использованием веб-страниц с программой обучения, другими статическими материалами и чат-сессиями, обеспечивающими интерактивное общение.
\item 4-ое поколение. Интерактивность в реальном времени с программным обеспечением, видеокамерами, объединенной системой управления (И.М. Радченко, 2005).
\end{itemize}

Сегодня появляется необходимость и возможность говорить о медиа-обучении,то есть о массовом медиа-образовании. Его предпосылкой является развитие новых технологий, прежде всего – компьютерных. Одним из плацдармов, на котором можно эффективно и целенаправленно развернуть формирование и информационной, и медиа-культуры, являются дистанционные технологии образования средствами Интернета. Они могут послужить противовесом деструктивному воздействию идеологии социального конструкционизма в практике медиа (А.А. Калмыков, 2009).

Задачи и принципы медиа-образования пока не входят непосредственно в содержание образовательных программ. Иными словами, можно констатировать, что дистанционное образование является до сих пор неиспользованным ресурсом формирования информационной культуры. 

Системы дистанционного обучения в зарубежных странах имеют развитую инфраструктуру и широкую пользовательскую аудиторию. Так, например, Открытый университет Великобритании имеет 305 региональных центров в Великобритании и 42 в других странах. Испанский национальный университет дистанционного образования имеет 53 региональных центра в Испании и Латинской Америке. Канадский открытый университет имеет 4 региональных центра. Ферн Университет Германии имеет 60 региональных центров в Германии, Австрии, Голландии, Венгрии, Польше. Открытый университет Израиля располагает более чем 100 региональными центрами. Национальный технологический университет США использует для обучения более 300 площадок на базе 46 вузов США (Н.А. Корнеева, 2007). 

В США интенсивно развивается предоставление интерактивных онлайн-курсов. К 2009 году более 5,6 миллионов студентов приняло участие в прохождении хотя бы одного онлайн-курса. Около 30\% процентов студентов получающих высшее образование участвуют в дистанционном обучении. Использование систем дистанционного обучения привело к росту количества студентов, получающих высшее образование на два процента (I.E. Allen, 2010).

В России системы дистанционного обучения имеют менее развитую инфраструктуру и находятся на этапе становления. Проблема дистанционного образования представляется особенно важной для России с ее огромной территорией, неравномерной плотностью населения и размещением вузовских центров (Г.В. Можаева 2000).

Актуальность развития дистанционного образования на территории Российской Федерации обуславливается следующими тенденциями:

\begin{itemize}
\item Россия испытывает на себе влияние процессов глобализации, информатизации образовательной среды. Как результат - подписание Болонского соглашения и следующие за ним демократизация, открытость образования, внедрение дистанционных технологий обучения.
\item с начала 90-х годов в России наблюдается недостаток востребованных рынком специалистов. Изменения в экономической сфере побуждают вузы к открытию новых специальностей, стимулируют развитие негосударственного сектора образования, более мобильного и адаптированного к запросам потребителя, в том числе по формам и технологиям обучения.
\item нарастает конкуренция вузов за потребителя, вызванная, с одной стороны, появлением наряду с государственными вузами негосударственных, с другой стороны, ухудшением демографической ситуации (Н.А. Корнеева, 2007).
\end{itemize}

Для улучшения качества дистанционного обучения необходимо восстановить полную структуру системы обучения, добавив в нее повсеместно отсутствующие в России отделы преподавания и разработки учебников. Крайне важно привлечь в указанные отделы наиболее опытных и квалифицированных преподавателей, которые бы занимались только своими прямыми обязанностями (В.И. Левин, 2005). 

Достоинствами систем дистанционного образования являются:

\begin{itemize}
\item технологичность - обучение с использованием современных программных и технических средств делает электронное образование более эффективным; 
\item доступность и открытость обучения - возможность учиться удалено от места обучения, не покидая свой дом или офис; 
\item свобода и гибкость - появляются новые возможности для выбора курса обучения. Очень легко выбрать несколько курсов из разных университетов, из разных стран. 
\item индивидуальность систем дистанционного обучения. Дистанционное обучение носит более индивидуальный характер обучения, более гибкое, обучающийся сам определяет темп обучения, может возвращаться по несколько раз к отдельным урокам, может пропускать отдельные разделы и т.д. 
\item внедрение дистанционного обучения уменьшает нервозность студентов при сдаче зачета или экзамена. 
\end{itemize}

Недостатками систем дистанционного образования являются:

\begin{itemize}
\item отсутствие прямого очного общения между студентами и преподавателем;
\item необходимость в персональном компьютере и доступе в Интернет;
\item проблема аутентификации пользователя при проверке знаний;
\item Высокая трудоемкость разработки курсов дистанционного обучения (Ю.А. Айзина, 2009). 
\end{itemize}

Инфраструктура системы дистанционного образования включает в себя центр ДО (ЦДО) базового образовательного учреждения и периферийные центры ДО (ПЦДО), связывающие их телекоммуникационные каналы и размещенные в центрах информационные ресурсы.

Построение распределенной образовательной среды позволяет снять ряд проблем, связанных с доступом к учебной информации. Создание локальных копий Web-ресурсов, их репликация и синхронизация позволяют обеспечить доступ средствами Интранет и значительно уменьшить Интернет-трафик (В.М. Вымятнин, 2000).

Технологическая компонента системы дистанционного обучения позволяет предоставлять учебные материалы в не только в текстовых форматах, но и с использованием мультимедиа (L. A. Schlosser, 2009). Данный формат представления информации позволяет пользователю системы лучше усваивать учебный материал (M.G.  Moore, 2011).

Согласно теории двойного кодирования информация воспринимается по одному из двух обычно независимых каналов. Один канал воспринимает устную информацию типа текста или аудио. Другой канал воспринимает невербальные изображения типа иллюстраций и звуков в окружающей среде. Информация может восприниматься через оба канала одновременно. Информация, переработанная таким образом, называется референциальной и имеет суммарный эффект на запоминание. Усваивание происходит лучше, когда информация воспринимается через два канала, чем через один канал. Референциальная обработка может производить этот суммарный эффект, потому, что обучающийся создает больше познавательных путей, по которым можно следовать, чтобы отыскать нужную информацию. Использование аудио с видео может задействовать множественные мозговые процессы, приводя к лучшему запоминанию (И.М. Радченко, 2005).

Дистанционное обучение, индивидуализированное по своей сути, не должно вместе с тем исключать возможностей коммуникации не только с преподавателем, но и с другими учащимися, сотрудничества в процессе разного рода познавательной и творческой деятельности. Проблемы социализации оказываются весьма актуальны при дистанционном обучении. 

Способами повышения качества систем дистанционного обучения и решения основных проблем при работе с ними являются: 

\begin{itemize}
\item стимулирование разработки качественных информационных ресурсов,
\item облегчение процесса публикации материалов, 
\item информирование образовательного сообщества о новых материалах,
\item привлечение авторитетных источников, 
\item механизмы рецензирования (А.Д Иванников, 2003).
\end{itemize}

Одним из перспективных способов повышения качества системы дистанционного обучения является использование в ней семантических технологий.



\section{Мировые тренды применения онтологий и семантических технологий в образовательном процессе} \label{sect1_3}

Использование онтологий в образовательном процессе позволяет получить ряд преимуществ, таких как:
\begin{itemize}
\item обмен информацией между системами обучения, 
\item предоставления платформ для повторного использования учебных объектов,
\item реализация интеллектуальной и персонализированной поддержки студента.
\end{itemize}

Использование онтологии в образовательной системе делает её более гибкой в отношении построения образовательного процесса и позволяет выгоднее использовать связи между данными (R. Wilson, 2004).

Одним из примеров онтологии предназначенной для использования в образовательном процессе является онтология AIISO (Academic Institution Internal Structure Ontology). Данная онтология предоставляет классы и свойства для описания организационной структуры образовательного учреждения и его учебных программ. AIISO позволяет описывать курсы, модули, факультеты, исследовательские группы и другие структурные объекты образовательного учреждения (R. Styles, 2008). 

Формат RDF позволяет производить хранение и взаимодействия различных наборов данных и спецификаций. В образовательном процессе одной из проблем стоит проблема внедрения новых образовательных спецификаций (N. Henze, 2004). При использовании формата RDF в образовательных приложениях снижает сложность внедрения новых спецификаций к уже существующим (K. Möller, 2010). Данный процесс сводится лишь к добавлению новой RDF схемы способной взаимодействовать с уже существующими. Такая возможность является очень важной на фоне необходимой адаптации образовательных технологий (M. Nilsson, 2001). 

Использование формата RDF и Semantic Web позволяет описывать весь спектр учебных объектов (A. Bouzeghoub, 2004). Различные домены и наборы данных, связанные между собой позволяют описывать:

\begin{itemize}
\item организационную структуру образовательного учреждения,
\item персоналии студентов и преподавателей,
\item учебные программы, курсы и модули,
\item механизмы проверки знаний (тесты, интерактивные стенды),
\item предметную терминологию,
\item вспомогательные данные (литература, мультимедиа, публикации).
\end{itemize}

Использование высокоуровневых стандартизированных образовательных онтологий позволяет разрабатываемой онтологии быть задействованной в различных образовательных системах (R. Bansal, 2012). 

Образовательная система, использующая технологию Semantic Web, предоставляет быстрый поиск по предметным терминам, что позволяет студенту получить полную информацию об интересующем его термине в мультимедиа материалами и ссылками на другие источники и термины в различных форматах. Данный подход делает содержание учебных объектов более полным и структурированным. Хранение данных в образовательной системе с использованием принципов Linked Data позволяет реализовывать механизмы автоматической актуализации содержания учебных объектов (P. Mohan, 2003). 

Структурированные учебные объекты и возможность персонализации образовательной системы с помощью принципов Linked Data делает  данную систему более гибкой (M. Nilsson, 2002). Преподаватели имеют возможность создания, редактирования и публикации собственных курсов. Данные курсы могут быть использованы в различных образовательных системах. Пользователь образовательной системы имеет возможность настройки структуры предоставляемых ему данных, в зависимости от собственных предпочтений (R. Koper, 2004).  

Персонализированная образовательная система, основанная на Semantic Web, также обладает сильным аналитическим потенциалом.  Собирая статистические данные по деятельности пользователей в системе, аналитические модули позволяют оценить качество курса, тестов и лекций (L. Aroyo, 2004). Данная статистика получается путем рецензирования пользователями учебных материалов, их посещаемости и популярности. Аналитические механизмы позволяют производить анализ качества учебных курсов и на внутрисистемном уровне, предоставляя преподавателям информацию о степени актуальности модулей курса и степени покрытия курса тестами (J. Puustjärvi, 2004).  

Помимо основной функции системы дистанционного обучения – образовательного процесса  пользователя системы через взаимодействие с преподавателем и учебным материалом существуют другие направления в дистанционном образовательном процессе, в которых использования технологий Linked Data и Semantic Web предоставляют ряд преимуществ. При организации взаимодействия между пользователями Semantic Web позволяет рядовым пользователям обмениваться и публиковать свои учебные материалы. Онтологические модели позволяют организовывать не только образовательный процесс, но и исследовательский процесс на базе единой системы дистанционного обучения. Связывания данных направлений реализуется с помощью публикации данных по принципам Linked Data (S. White, 2013). 

Развитие образовательных систем не стоит на месте и развивается вместе с развитием технологий других отраслей. Образовательный процесс постепенно переходит из университетских аудиторий в виртуальное пространство. Развитие технологи коммуникаций позволяет организовывать виртуальные лекции и конференции. Студенты слушают лекции и проходят стажировки в компаниях, не выходя из дома (E. Masie, 2012). Рост количества мобильных устройств среди студентов позволяет им получать учебные материалы, где бы они не находились. Университетские и государственные библиотеки постепенно размещают цифровые копии своих материалов в сети. Пользовательская аудитория систем дистанционного образования состоит не только из студентов университетов, но и пользователей с высшим образованием желающих расширить свои знания в той или иной области (D.G. Oblinger, 2010). 

Данные факты определяют необходимость тесного взаимодействия информационных технологий и процесса образования. Современная дистанционная система должна быть ориентирована не только на учащихся определенного университета, а предоставлять учебные курсы всем желающим. Пользователь, не привязанный к определенному университету должен иметь возможность выбора интересующего его направления обучения. 

Поддержка дистанционной системы обучения крупного масштаба, рассчитанной на большое количество пользователей, не относящихся к какому-либо университету, требует огромных ресурсов. Главными проблемами данной системы являются: 

\begin{itemize}
\item наполнение системы курсам и учебными материалами,
\item поддержка актуальности учебных материалов, 
\item оценка покрытия тестами учебного курса,
\item организация проверки знаний обучаемых.
\end{itemize}

Использование онтологий и технологий Semantic Web и Linked Data в системе дистанционного обучения позволяет решить данные проблемы. Данный подход автоматизирует большинство процессов, позволяя производить следующие действия с системой:

\begin{itemize}
\item автоматический анализ учебных материалов;
\item связывание и интеграция учебных материалов внешних источников в систему;
\item автоматическая обработка материалов проверки знаний пользователей с дальнейшей публикацией результатов. 
\end{itemize}

Таким образом, создание открытой системы дистанционного обучения на основе технологий Semantic Web и Linked Data, позволяет предоставить обучающимся более гибкий, структурированный и конкретизированный процесс обучения. 


\clearpage
