\chapter*{Введение}							% Заголовок
\addcontentsline{toc}{chapter}{Введение}	% Добавляем его в оглавление

\textbf{Актуальность темы исследования.}
Популярность и стремительное развитие систем электронного обучения и платформ для массовых открытых онлайн-курсов MOOC (Massive Open Online Course) в последнее время привели к появлению огромного количества образовательных ресурсов, открытых, но практически никак не связанных между собой. Так крупнейший проект в области MOOC - Coursera насчитывает более 1000 онлайн-курсов, предоставленных более чем 100 университетами и организациями. Другими крупными MOOC порталами являются проекты EdX и Udacity, суммарная аудитория которых превышает 4 миллиона пользователей. В России проект <<ИНТУИТ>> позволяет пользователям бесплатно проходить обучение по образовательным программам, многие из которых касаются информационных технологий. Проект <<ИНТУИТ>> предоставляет пользователям более 800 онлайн-курсов. Проект <<Лекториум>> занимается созданием и размещением открытых учебных курсов в формате видеолекций. Курсы проекта подготовлены ведущими ВУЗами и организациями России. Всего в проекте насчитывается более 4000 часов видеолекций. Отдельный вклад в увеличение количества образовательных ресурсов в сети Интернет внесли электронные библиотеки. Так Британская Библиотека предоставляет информацию о более чем 3,5 миллионах публикаций, книг и монографий в формате открытых данных с помощью проекта BNB (British National Bibliography). 

В области разработки систем обучения и автоматизации образовательных процессов возникает необходимость в сборе, агрегации, гармонизации и повторном использовании учебных материалов различных образовательных ресурсов в контексте одной системы электронного обучения. В настоящее время повторное использование учебных материалов сетевых образовательных ресурсов является одним из наиболее перспективных подходов для разработки систем электронного обучения. Разработка методик агрегации данных позволит создавать распределенные системы электронного обучения, использующие учебные материалы из различных образовательных ресурсов университетов, библиотек и организаций.

Одним из методов реализации повторного использования учебных материалов в системах электронного обучения является применение семантических технологий. Технологии семантических сетей (Semantic Web) и связанных данных (Linked Data) позволяют системам обмениваться данными в сети с использованием онтологий и уникальных идентификаторов ресурсов URI (Uniform Resource Identifier). Системы, используя данные технологии, могут интегрировать и адаптировать данные из сторонних источников. Семантические технологии широко используются в образовательных ресурсах в большинстве развитых стран. Наиболее известным проектом в области связанных данных является инициатива Linked Universities. Linked Universities является альянсом европейских университетов распространяющих свои курсы и учебные материалы в формате Linked Data. Другим успешным примером использования семантических технологий в системе электронного обучения является Open University. Open University является исследовательским университетом электронного обучения с более чем 240 тысячами студентов. Семантические технологии используются в прикладных образовательных системах. Одной из таких систем является система SlideWiki - система управления обучением, позволяющая составлять курсы на основе презентаций. Платформа реализует возможность повторного использования данных созданных курсов при составлении нового учебного курса.

Автоматизация сбора, агрегации и гармонизации учебных материалов сетевых образовательных ресурсов в системе электронного обучения позволяет поддерживать содержание образовательного процесса в актуальном состоянии. Пользователям системы предоставляются как одобренные и проверенные учебные материалы, так и материалы из открытых, общедоступных источников. При росте объемов учебных материалов и автоматизации наполнения системы электронного обучения возникает необходимость в контроле качества и целостности учебных курсов и программ. Семантические технологии позволяют формально описывать структуру и связи объектов образовательного процесса. На основе этих связей может быть разработана методика анализа качества учебного курса. 

Действия студентов в системе электронного обучения могут быть связаны с описанными формально объектами образовательного процесса. Разработка методики анализа деятельности пользователей на основе неявных связей с объектами образовательного процесса позволит преподавателям и авторам курсов получить отклик от студентов. На основе анализа качества, целостности курсов, а так же на основе откликов студентов, авторы курсов и преподаватели могут вносить коррективы в образовательный процесс с целью повышения качества обучения. Аналитическая информация деятельности студентов в системе позволит студентам производить мониторинг процесса обучения.     

\textbf{Целью диссертационной работы} является исследование и разработка методов агрегирования и анализа образовательных данных в системе электронного обучения с использованием семантических технологий.

Для достижения поставленной цели необходимо было решить следующие \textbf{задачи}:
\begin{enumerate}
 \item Разработать онтологические модели для описания учебных материалов, тестов, предметных областей учебных дисциплин, результатов и процесса обучения студентов, метрик оценки знаний и рейтинга студентов;
 \item Разработать методику автоматического агрегирования образовательных данных в системе электронного обучения с использованием методов естественной обработки языка, логического вывода, технологий Semantic Web и Linked Data;
 \item Разработать метод анализа качества и полноты образовательных материалов, основанный на семантических связях в онтологических моделях;
  \item Разработать новый подход к автоматизированной оценке результатов обучения и построению рейтингов студентов системы электронного обучения с использованием статистических методов и семантических технологий;
  \item Создать программные модули для системы электронного обучения на основе разработанных методов и проанализировать полученные результаты на практике.  
 \end{enumerate}

\textbf{Объектом исследования} является структура и логические отношения между объектами образовательного процесса, предметных областей учебных дисциплин и образовательных ресурсов.  

\textbf{Предметом исследования} является алгоритмическое обеспечение, предназначенное для автоматизации процессов наполнения системы электронного обучения учебными материалами, процессов анализа качества учебных материалов и знаний студентов. 

\textbf{Научная новизна.}
\begin{enumerate}
 \item Впервые разработана онтологическая модель, описывающая образовательный процесс, учебные материалы, предметные области, действия студентов и отношения между данными объектами. 
 \item Впервые разработан метод оценки качества и полноты учебного курса на основе анализа неявных семантических связей.
 \item Впервые предложен подход к автоматизированному анализу знаний и рейтингов студентов системы электронного обучения с использованием статистических методов и семантических технологий.
\end{enumerate}

\textbf{Основные положения, выносимые на защиту.}
\begin{enumerate}
 \item Модульная онтология, описывающая структуру образовательного процесса, структуру предметной области учебной дисциплины, структуру тестов, действия и результаты студентов в системе электронного обучения.
 \item Методика наполнения образовательной онтологии и агрегации учебных материалов в системе электронного обучения.
 \item Метод анализа качества и полноты учебного курса;
 \item Алгоритм анализа знаний и рейтинга студентов системы электронного обучения.
 \end{enumerate}


\textbf{Практическая значимость.}
\begin{enumerate}
 \item Разработанная модульная онтология опубликована в сети с зарегистрированными идентификаторами PURLs (Persistent Uniform Resource Locators) и может быть использована для разработки систем электронного обучения.
 \item Программная реализация системы электронного обучения, использующая разработанные модели и алгоритмы, была внедрена на открытой площадке на кафедре проектирования и безопасности компьютерных систем Университета ИТМО (ecole.ifmo.ru). 
 \item Созданы на основе разработанных моделей и алгоритмов и используются на практике онлайн курсы 
<<Интеллектуальные системы>>, <<Физика>>, <<Теория графов>> и <<Аналитическая геометрия и линейная алгебра>>.
 \end{enumerate}
 
 
%\textbf{Достоверность} изложенных в работе результатов обеспечивается ...


\textbf{Апробация работы.}
Основные результаты диссертационной работы докладывались и обсуждались на следующих конференциях:
International Conference on Knowledge Engineering and Semantic Web (Россия, Санкт-Петербург, 2013),
11th Extended Semantic Web Conference (Греция, Аниссарас, 2014), International Conference on Knowledge Engineering and Semantic Web (Россия, Казань, 2013), The 13th International Semantic Web Conference (Италия, Рива-дель-Гарда, 2014), 16th Conference of Open
Innovations Association FRUCT (Финляндия, Оулу, 2014).

\textbf{Публикации.} Основные результаты по теме диссертации изложены в 6 печатных изданиях, 1 из которых изданы в журналах, рекомендованных ВАК, 5 в тезисах докладов.

\textbf{Объем и структура работы.} Диссертация состоит из введения, четырех глав, заключения и приложения. Полный объем диссертации \textbf{110} страниц текста с \textbf{20} рисунками и \textbf{6} таблицами. Список литературы содержит \textbf{65} наименование.

\clearpage